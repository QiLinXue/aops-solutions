%%%%%%%%%
% Setup %
%%%%%%%%%
\hfuzz=100pt 

\SetupExSheets{
  question/post-body-hook = {%
    \hyperlink{sol:\CurrentQuestionID}{ (View Solution)}
  },
  solution/pre-hook = {
    \hypertarget{sol:\CurrentQuestionID}{}%
  } ,
  solution/pre-body-hook = {%
    \hyperref[qu:\CurrentQuestionID]{ (View Question)}\par
  }
}

\SetupExSheets{
  counter-format = ch.qu,
  headings=runin
} 

%%%%%%%%%%%%%%%%%%%%%%%%%%%%%%%%%%%%%%%%%%%%%%%%%%%%%%%%%%%%%%%%%%%%
%%%%%%%%%%%%%%%%%%%%%%%%%%%%%%%%%%%%%%%%%%%%%%%%%%%%%%%%%%%%%%%%%%%%
%%%%%%%%%%%%%%%%%%%%%%%%%%%%%%%%%%%%%%%%%%%%%%%%%%%%%%%%%%%%%%%%%%%%
%%%%%%%%%%%%%%%%%%%%%%%%%%%%%%%%%%%%%%%%%%%%%%%%%%%%%%%%%%%%%%%%%%%%
%%%%%%%%%%%%%%%%%%%%%%%%%%%%%%%%%%%%%%%%%%%%%%%%%%%%%%%%%%%%%%%%%%%%
%%%%%%%%%%%%%%%%%%%%%%%%%%%%%%%%%%%%%%%%%%%%%%%%%%%%%%%%%%%%%%%%%%%%

\section{Problems}

%%%%%%%%%%%%%%%%%%%%%%%%%%%%%%%%%%
%%%%%%%%%%%%%%%%%%%%%%%%%%%%%%%%%%
%%%%%%%%%%%%%%%%%%%%%%%%%%%%%%%%%%

\begin{question}

Evaluate the product $(\log_2{3})(\log_3{4})(\log_4{5})(\log_5{6})(\log_6{7})(\log_7{8})$

\end{question}

\begin{solution}

Use equation (1.4) we can note that if we have a product of logs and keep the bases the same, we can shift the arguments around however much we want. If we move all the arguments one to the right, and move the final argument to the first one, we get

$$(\log_2{8})(\log_3{3})(\log_4{4})(\log_5{5})(\log_6{6})(\log_7{7})$$

Since $\log_n{n} = 1$ we can cancel everything and get our answer of \boxed{3} 

\end{solution}

%%%%%%%%%%%%%%%%%%%%%%%%%%%%%%%%%%
%%%%%%%%%%%%%%%%%%%%%%%%%%%%%%%%%%
%%%%%%%%%%%%%%%%%%%%%%%%%%%%%%%%%%

\begin{question}

If $\log(36) = a$ and $\log(125) = b$, express $\log(\frac{1}{12})$ in terms of $a$ and $b$. (MAO 1992)

\end{question}

\begin{solution}

First we note that $36$ and $125$ are both a perfect square and cube, respectively. Therefore, we can simplify these equations by writing them with an exponent: 
$\log(36) = \log(6^2) = 2\log(6)$ and $\log(125) = \log(5^3) = 3\log(5)$

Start with $\log(\frac{1}{12})$ and note it is equal to $\log(1)-\log(12) = -\log(12)$. We know from the first step we need to somehow include both $\log(5)$ and $\log(6)$, therefore we need to include these expressions in our solution somewhere...

After some experimentation, we find that $-\log(12) = -\log(\frac{6 \times 10}{5})$ which is equal to:

$$-(\log(6 \times 10) - \log(5) = -(\log(6)+\log(10)-\log(5)$$
$$-(\frac{a}{2}+1-\frac{b}{3})$$

The answer is \boxed{\frac{b}{3}-\frac{a}{2}-1}

\end{solution}

%%%%%%%%%%%%%%%%%%%%%%%%%%%%%%%%%%
%%%%%%%%%%%%%%%%%%%%%%%%%%%%%%%%%%
%%%%%%%%%%%%%%%%%%%%%%%%%%%%%%%%%%

%%%%%%%%%%%%%%%%%%%%%%%%%%%%%%%%%%%%%%%%%%%%%%%%%%%%%%%%%%%%%%%%%%%%
%%%%%%%%%%%%%%%%%%%%%%%%%%%%%%%%%%%%%%%%%%%%%%%%%%%%%%%%%%%%%%%%%%%%
%%%%%%%%%%%%%%%%%%%%%%%%%%%%%%%%%%%%%%%%%%%%%%%%%%%%%%%%%%%%%%%%%%%%
%%%%%%%%%%%%%%%%%%%%%%%%%%%%%%%%%%%%%%%%%%%%%%%%%%%%%%%%%%%%%%%%%%%%
%%%%%%%%%%%%%%%%%%%%%%%%%%%%%%%%%%%%%%%%%%%%%%%%%%%%%%%%%%%%%%%%%%%%
%%%%%%%%%%%%%%%%%%%%%%%%%%%%%%%%%%%%%%%%%%%%%%%%%%%%%%%%%%%%%%%%%%%%

\newpage
\section{Solutions}


\printsolutions