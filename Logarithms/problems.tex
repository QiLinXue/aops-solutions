%%%%%%%%%
% Setup %
%%%%%%%%%
\hfuzz=100pt 

\SetupExSheets{
  question/pre-body-hook = {%
    \hyperlink{sol:\CurrentQuestionID}{ (View Solution)}
  },
  solution/pre-hook = {
    \hypertarget{sol:\CurrentQuestionID}{}%
  } ,
  solution/pre-body-hook = {%
    \hyperref[qu:\CurrentQuestionID]{ (View Question)}\par
  }
}

\SetupExSheets{
  counter-format = ch.qu,
  headings=runin
} 

%%%%%%%%%%%%%%%%%%%%%%%%%%%%%%%%%%%%%%%%%%%%%%%%%%%%%%%%%%%%%%%%%%%%
%%%%%%%%%%%%%%%%%%%%%%%%%%%%%%%%%%%%%%%%%%%%%%%%%%%%%%%%%%%%%%%%%%%%
%%%%%%%%%%%%%%%%%%%%%%%%%%%%%%%%%%%%%%%%%%%%%%%%%%%%%%%%%%%%%%%%%%%%
%%%%%%%%%%%%%%%%%%%%%%%%%%%%%%%%%%%%%%%%%%%%%%%%%%%%%%%%%%%%%%%%%%%%
%%%%%%%%%%%%%%%%%%%%%%%%%%%%%%%%%%%%%%%%%%%%%%%%%%%%%%%%%%%%%%%%%%%%
%%%%%%%%%%%%%%%%%%%%%%%%%%%%%%%%%%%%%%%%%%%%%%%%%%%%%%%%%%%%%%%%%%%%
\newpage
\section{Problems}

%%%%%%%%%%%%%%%%%%%%%%%%%%%%%%%%%%
%%%%%%%%%%%%%%%%%%%%%%%%%%%%%%%%%%
%%%%%%%%%%%%%%%%%%%%%%%%%%%%%%%%%%

\begin{question}

Evaluate the product $(\log_2{3})(\log_3{4})(\log_4{5})(\log_5{6})(\log_6{7})(\log_7{8})$

\end{question}

\begin{solution}

Use equation (1.4) we can note that if we have a product of logs and keep the bases the same, we can shift the arguments around however much we want. If we move all the arguments one to the right, and move the final argument to the first one, we get

$$(\log_2{8})(\log_3{3})(\log_4{4})(\log_5{5})(\log_6{6})(\log_7{7})$$

Since $\log_n{n} = 1$ we can cancel everything and get our answer of \boxed{3} 

\end{solution}

%%%%%%%%%%%%%%%%%%%%%%%%%%%%%%%%%%
%%%%%%%%%%%%%%%%%%%%%%%%%%%%%%%%%%
%%%%%%%%%%%%%%%%%%%%%%%%%%%%%%%%%%

\begin{question}

If $\log(36) = a$ and $\log(125) = b$, express $\log(\frac{1}{12})$ in terms of $a$ and $b$.

\end{question}

\begin{solution}

First we note that $36$ and $125$ are both a perfect square and cube, respectively. Therefore, we can simplify these equations by writing them with an exponent: 
$\log(36) = \log(6^2) = 2\log(6)$ and $\log(125) = \log(5^3) = 3\log(5)$

Start with $\log(\frac{1}{12})$ and note it is equal to $\log(1)-\log(12) = -\log(12)$. We know from the first step we need to somehow include both $\log(5)$ and $\log(6)$, therefore we need to include these expressions in our solution somewhere...

After some experimentation, we find that $-\log(12) = -\log(\frac{6 \times 10}{5})$ which is equal to:

$$-(\log(6 \times 10) - \log(5) = -(\log(6)+\log(10)-\log(5)$$
$$-(\frac{a}{2}+1-\frac{b}{3})$$

The answer is \boxed{\frac{b}{3}-\frac{a}{2}-1}

\end{solution}

%%%%%%%%%%%%%%%%%%%%%%%%%%%%%%%%%%
%%%%%%%%%%%%%%%%%%%%%%%%%%%%%%%%%%
%%%%%%%%%%%%%%%%%%%%%%%%%%%%%%%%%%

\begin{question}

At which point(s) do $y=2\log(x)$ and $y=\log(2x)$ intersect?

\end{question}

\begin{solution}

To start, let's set both equations equal to each other: $2\log(x)=\log(2x)$. Notice that 
$2\log(x)=\log(x^2)$ We can simplify both sides:

$$10^{\log(x^2)}=2^{\log(2x)} \rightarrow x^2 = 2x $$

Solving this quadratic equation gives $x=2$ and $x=0$ Since $\log(0)$ is undefined,
\boxed{x=2} is the only interesection point

\newpage
\end{solution}

%%%%%%%%%%%%%%%%%%%%%%%%%%%%%%%%%%
%%%%%%%%%%%%%%%%%%%%%%%%%%%%%%%%%%
%%%%%%%%%%%%%%%%%%%%%%%%%%%%%%%%%%
\begin{question}

Find all solutions of:

$$x^{\log(x)} = \frac{x^3}{100}$$

\end{question}

\begin{solution}
We can take the base-x logarithm of both sides and using theorem (1.1) and (1.5):

$$
  \log_x(x^{\log(x)}) = \log_x(\frac{x^3}{100}) 
  \rightarrow 
  \log(x)\log_x(x) = \log_x(\frac{x^3}{10^2})
$$

$$
  \log_{10}(x) = \log_x(x^3)-\log_x(10^2)
  \rightarrow
  \log_{10}(x) = 3-\frac{2\log_{10}(10)}{\log_{10}(x)}
  \rightarrow
  \log_{10}(x) = 3-\frac{2}{\log_{10}(x)}
$$

Substituting $u=\log_{10}(x)$ gives:

$$ u = 3-\frac{2}{u} $$

Solving this quadratic equation gives us $u\in\{1,2\}$ and thus \boxed{x\in\{10, 100\}}

\end{solution}

%%%%%%%%%%%%%%%%%%%%%%%%%%%%%%%%%%
%%%%%%%%%%%%%%%%%%%%%%%%%%%%%%%%%%
%%%%%%%%%%%%%%%%%%%%%%%%%%%%%%%%%%
\begin{question}

If $a > 1$, $b > 1$, and $p = \frac{\log_b(\log_b(a))}{\log_b(a)}$, then find $a^p$ in simplest form

\end{question}

\begin{solution}

First, we can apply the change of base formula:

$$ 
   p = \frac{\log_b(\log_b(a))}{\log_b(a)} \rightarrow
   p = \log_a(\log_b(a))
$$

Raise a to the power of both sides and we get: \boxed{a^p = \log_b(a)}

\end{solution}

%%%%%%%%%%%%%%%%%%%%%%%%%%%%%%%%%%
%%%%%%%%%%%%%%%%%%%%%%%%%%%%%%%%%%
%%%%%%%%%%%%%%%%%%%%%%%%%%%%%%%%%%
\begin{question}

If one uses only the information $10^3 = 1000$, $10^4 = 10000$, $2^{10} = 1024$, $2^{11}=2048$,
$2^{12}=4096$,  $2^{13}=8192$, what are the largest $a$ and $b$ such that one can prove
$a<\log_{10}(2)<b$?

\end{question}

\begin{solution}

First, notice that $2^10 \approx 10^3$ so maybe we can use this somehow in our answer? First, we can rewrite our inequality:

$$
a<\log_{10}(2)<b \rightarrow 10^a < 2 < 10^b
$$

Now, raise everything to the power of 10 so the middle number could be $2^{10} = 1024 \approx 1000$. This would lead to:

$$
10^{10a} < 1024 < 10^{10b}
$$

We know that the two closest powers of 10 near 1024 is $10^3$ and $10^4$. Thus: $10a=3$ and $10b=4$. Solving these two equations gives \boxed{a=\frac{3}{10},b=\frac{2}{5}}

\end{solution}

%%%%%%%%%%%%%%%%%%%%%%%%%%%%%%%%%%
%%%%%%%%%%%%%%%%%%%%%%%%%%%%%%%%%%
%%%%%%%%%%%%%%%%%%%%%%%%%%%%%%%%%%
\begin{question}

Simplify $\frac{1}{\log_3(x)} + \frac{1}{\log_4(x)} + \frac{1}{\log_5(x)}$

\end{question}

\begin{solution}
We can perform a change of base to base-x on each individual term to bring the logarithm in the numerator. Note that:

$$\frac{1}{\log_n(x)} = \frac{\log_n(n)}{\log_n(x)} = \log_x(n)$$

\newpage

Thus:

$$
\frac{1}{\log_3(x)} + \frac{1}{\log_4(x)} + \frac{1}{\log_5(x)} =
\log_x(3) + \log_x(4) + \log_x(5) =
\log_x(3 \cdot 4 \cdot 5) =
\log_x(60)
$$

The answer is \boxed{\log_x(60)}

\end{solution}

%%%%%%%%%%%%%%%%%%%%%%%%%%%%%%%%%%
%%%%%%%%%%%%%%%%%%%%%%%%%%%%%%%%%%
%%%%%%%%%%%%%%%%%%%%%%%%%%%%%%%%%%
\begin{question}

Given that $\log_{10}(2) \approx 0.3010$, how many digits are in $5^44$?

\end{question}

\begin{solution}

To find the number of digits in a number, we have to take the base-10 logarithm of the number, and round it up. Try to convince yourself why this is true. The proof is trivial and will be left as an exercise to the reader.
\newline\newline
Therefore, if $x = 5^{44}$, then the number of digits of x will be equal to the ceiling of $\log_{10}(x)$. The question also gave us the logarithm of 2, so that has to come into play somewhere... Note that  $5^{44} \times 2^{44} = 2^{44} \times x$. We can collapse the left side to $10^{44}$ and take the logarithm of both sides such that:

$$
  \log_{10}(10^{44}) = \log_{10}(2^{44} \times x) \rightarrow
  44 = 44\log_{10}(2) + \log_{10}(x) \rightarrow
  44 = 44(0.3010) + \log_{10}(x)
$$

Solving for $\log_{10}(x)$ gives 30.756, rouding it up gives \boxed{31} digits.
\end{solution}

%%%%%%%%%%%%%%%%%%%%%%%%%%%%%%%%%%
%%%%%%%%%%%%%%%%%%%%%%%%%%%%%%%%%%
%%%%%%%%%%%%%%%%%%%%%%%%%%%%%%%%%%
\begin{question}

If $\log_8(3) = P$ and $\log_3(5) = Q$, express $\log_{10}(5)$ in terms of P and Q.

\end{question}

\begin{solution}

Let us start with $\log_8(3)$ and $\log_3(5)$ and try to work our way towards $\log_{10}(5)$ by applying a series of transformations. If we multiply them using equation (1.4):

$$
   \log_8(3) \times \log_3(5) = \log_8(5) \times \log_3(3) = \log_8(5) = PQ
$$

Notice that we have the correct argument, but the wrong base. If we are able to figure out what $\log_8(10)$ is, we can do a change of base. Can we use a bit of wishful thinking and try to create that from $\log_8(5)$? Yes we can!

$$
   \log_8(10) = \log_8(2 \cdot 5) = \log_8(2) + \log_8(5) = \frac{1}{3} + PQ
$$

Now we can apply a change of base:

$$
   \frac{\log_8(5)}{\log_8(10)} = \log_{10}(5) = \frac{PQ}{\frac{1}{3}+PQ}
$$

We can simplify the answer slightly to \boxed{\frac{3PQ}{3PQ+1}}

\end{solution}

%%%%%%%%%%%%%%%%%%%%%%%%%%%%%%%%%%
%%%%%%%%%%%%%%%%%%%%%%%%%%%%%%%%%%
%%%%%%%%%%%%%%%%%%%%%%%%%%%%%%%%%%
\begin{question}

Suppose that $p$ and $q$ are positive numbers for which

$$
   \log_9(p) = \log_{12}(q) = log_{16}(p+q)
$$

What is the value of $\frac{q}{p}$ ?
\end{question}

\begin{solution}

An equality that goes three ways? This reminds me of triangles! Or more specifically, similar triangles! Or even more specifically, comparing ratios. Can these logarithms be written in simple ratios? A change in base (1.5) will achieve just that:

$$
   \log_9(p) = \log_{12}(q) = log_{16}(p+q) \rightarrow
   \frac{\ln(p)}{\ln(9)} = \frac{\ln(q)}{\ln(12)} = \frac{\ln(p+q)}{\ln(16)}
$$

Cross multiplying them, we get:

$$
  \ln(p)\ln(12) = \ln(q)\ln(9)
$$

and

$$
  \ln(12)\ln(p+q) = \ln(q)\ln(16)
$$

Adding these two equations up gives:

\begin{equation*}
  \begin{aligned}
  & \ln(12)(\ln(p)+\ln(p+q)) = \ln(q)(\ln(9)+\ln(16)) \\
  & \ln(12)(\ln(p(p+q))) = \ln(q)(\ln(144)) \\
  & \ln(12)(\ln(p(p+q))) = 2\ln(q)(\ln(12)) \\
  & \ln(p(p+q)) = \ln(q^2) \\
  & p^2 + pq = q^2
  \end{aligned}
\end{equation*}

Divide both sides by $pq$ and set the desired value of $\frac{q}{p}$ to x:

$$
   \frac{p^2}{pq} + \frac{pq}{pq} = \frac{q}{p} \rightarrow x + 1 + \frac{1}{x}
$$

Solving this quadratic gives the answer as: \boxed{\frac{1+\sqrt{5}}{2}}

\end{solution}

%%%%%%%%%%%%%%%%%%%%%%%%%%%%%%%%%%
%%%%%%%%%%%%%%%%%%%%%%%%%%%%%%%%%%
%%%%%%%%%%%%%%%%%%%%%%%%%%%%%%%%%%

\begin{question}

Given that $\log_{4n}(40\sqrt{3}) = \log_{3n}(45)$, find $n^3$

\end{question}
  
\begin{solution}

Let $ t = \log_{4n}(40\sqrt{3}) = \log_{3n}(45) $ such that $4^tn^t = 40\sqrt{3}$ and
$3^tn^t = 45$. Make a ratio out of these two equations:

$$
\begin{aligned}
& \frac{4^tn^t}{3^tn^t} = \frac{40\sqrt{3}}{45} \\
& \frac{4}{3}^t = \frac{8\sqrt{3}}{9} \\
& \frac{4}{3}^t = \frac{8}{3\sqrt{3}}
\end{aligned}
$$

We just need to solve for t. Under closer examination, we find that the numerator and denominator can both be multiplied by their respective square roots to get to their counterparts on the right side. Therefore, $t=\frac{3}{2}$ using the equation from earlier:

$$
4^\frac{3}{2}n^\frac{3}{2} = 40\sqrt{3} \rightarrow n^\frac{3}{2} = 5\sqrt{3} \rightarrow
n^3 = 25 \times 3 = 75
$$

The answer is \boxed{n^3 = 75}

\end{solution}

%%%%%%%%%%%%%%%%%%%%%%%%%%%%%%%%%%
%%%%%%%%%%%%%%%%%%%%%%%%%%%%%%%%%%
%%%%%%%%%%%%%%%%%%%%%%%%%%%%%%%%%%

\begin{question}

Suppose $a$ and $b$ are positive numbeers for which

$$
  \log_9(a) = \log_{15}(b) = \log_{25}(a+2b)
$$

What is the value of $\frac{b}{a}?$

\end{question}
    
\begin{solution}

The answer can be solved in the exact same steps as Exercise 1.10. Simplify perform a change in base,and create a syystem of two equations:

\begin{equation*}
  \begin{aligned}
    & \log_9(a) = \log_{15}(b) = \log_{25}(a+2b) \\
    & \frac{\ln(9)}{\ln(a)} = \frac{\ln(15)}{\ln(b)} = \frac{\ln(25)}{\ln(a+2b)} \\
    & \begin{cases}
      \ln(a)\ln(15) = \ln(b)\ln(9) \\
      \ln(a+2b)\ln(15) = \ln(b)\ln(25)
    \end{cases}
  \end{aligned}
\end{equation*}

Add both equations up to create:

\begin{equation*}
  \begin{aligned}
  & \ln(15)(\ln(a)+\ln(a+2b)) = \ln(b)(\ln(9)+\ln(25)) \\
  & \ln(15)(\ln(a(a+2b))) = \ln(b)(2\ln(3)+2\ln(5)) \\
  & \ln(15)(\ln(a(a+2b))) = 2\ln(b)\ln(15) \\
  & \ln(a(a+b)) = \ln(b^2) \\
  & a^2 + ab = b^2
  \end{aligned}
\end{equation*}

This results in the same answer as 1.10: \boxed{\frac{1+\sqrt{5}}{2}}

\end{solution}


%%%%%%%%%%%%%%%%%%%%%%%%%%%%%%%%%%
%%%%%%%%%%%%%%%%%%%%%%%%%%%%%%%%%%
%%%%%%%%%%%%%%%%%%%%%%%%%%%%%%%%%%

\begin{question}

If $60^a = 3$ and $60^b = 5$, then find

$$
  12^{\frac{1-a-b}{2-2b}}
$$

\end{question}
    
\begin{solution}
  
The trick to this question is to first rewrite the given in terms of a and b, then work backwards. We know that $a=\log_{60}(3)$ and $b=\log_{60}(5)$. Just looking at the exponent of 
$12^{\frac{1-a-b}{2-2b}}$ we have:

\begin{equation*}
  \begin{aligned}
  & \frac{1-a-b}{2-2b} = 
    \frac{\log_{60}(60)-\log_{60}(3)-\log_{60}(5)}
         {2(\log_{60}(60)-\log_{60}(5))}\\
  & \frac{\log_{60}(60/3/5)}{2\log_{60}(60/5)} = \frac{\log_{60}(4)}{2\log_{60}(12} \\
  & \frac{\log_{60}(4)}{\log_{60}(144)} = \log_{144}(4) = \log_{12}(2)
  \end{aligned}
\end{equation*}

Now after adding in the base:

$$
  12^{\log_{12}(2)} = 2
$$

The answer is \boxed{2}

\end{solution}

%%%%%%%%%%%%%%%%%%%%%%%%%%%%%%%%%%%%%%%%%%%%%%%%%%%%%%%%%%%%%%%%%%%%
%%%%%%%%%%%%%%%%%%%%%%%%%%%%%%%%%%%%%%%%%%%%%%%%%%%%%%%%%%%%%%%%%%%%
%%%%%%%%%%%%%%%%%%%%%%%%%%%%%%%%%%%%%%%%%%%%%%%%%%%%%%%%%%%%%%%%%%%%
%%%%%%%%%%%%%%%%%%%%%%%%%%%%%%%%%%%%%%%%%%%%%%%%%%%%%%%%%%%%%%%%%%%%
%%%%%%%%%%%%%%%%%%%%%%%%%%%%%%%%%%%%%%%%%%%%%%%%%%%%%%%%%%%%%%%%%%%%
%%%%%%%%%%%%%%%%%%%%%%%%%%%%%%%%%%%%%%%%%%%%%%%%%%%%%%%%%%%%%%%%%%%%

\newpage
\section{Solutions}


\printsolutions